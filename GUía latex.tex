\documentclass[11pt]{article}
\usepackage[T1]{fontenc}
\usepackage[utf8]{inputenc}
\usepackage{breqn} % corregir error en simbolos < y > con babel
\usepackage[spanish]{babel}
\usepackage{pstricks,pst-node,pst-text,pst-3d,pst-grad,pst-coil}
\usepackage{graphicx}
\usepackage{pst-fill}
\usepackage{color}
\bibliographystyle{plain}
\usepackage[colorlinks,urlcolor=blue]{hyperref}
\usepackage{amsmath}
%\usepackage{wrapfig}  % Pacote para colocar texto com figura
\usepackage{verbatim}
\usepackage{amsmath,amssymb}
\usepackage{makeidx}
\usepackage{hyperref}
\usepackage{times}
\usepackage{breqn} %slit equations
\usepackage{tikz}

%%%%%%%%%%%Color to equation important
\usepackage{amsmath,empheq,xcolor}
 \newcommand*\widefbox[1]{\fbox{\hspace{1em}#1\hspace{1em}}}
 \usepackage{xparse}% flexible defining of document commands
 \NewDocumentEnvironment{important}{O{red}O{}m}{%
   \setkeys{EmphEqEnv}{#3}%
   \setkeys{EmphEqOpt}{box=\widefbox,#2}%
  % \setkeys{EmphEqOpt}{box=\normalcolor\widefbox,#2}% recuadro negro
  
   \color{#1}%
   \EmphEqMainEnv}%
   {\endEmphEqMainEnv}
%%%%%%%%%%%Color to equation important


%
\textwidth 6.5in
\topmargin -0.5in
\textheight = 46.5\baselineskip
\oddsidemargin -.1in
\evensidemargin -.1in
%
\title{Proyecto 1 Ondas:     }
\author{Nombre}
\date{Fecha}

\begin{document}

\maketitle



%%%%%%%%%%%%%%%%%%%%%%%%%%%%%%%%%%%%%%%%%%%%%%%%%%%%%%%%%%%%%%%%%%%%%%%%
\section{Osciladores forzados amortiguados}
\subsection{Ecuación de movimiento}

Fuerza impulsadora periódica $F$:
\begin{align*}
F  \, = \,F_0 \,\cos\left( \omega_F \,t \right) 
\end{align*}
dónde, $\omega_F$ es la frecuencia angular de la fuerza externa. 

El símbolo $\&$ se usa para alinear las ecuaciones, al final de cada línea se usa  para que contínue en la siguiente línea. El comando $\nonumber$ se usa para no numerar la ecuación.
\begin{align}
&\sum \,F \, = m\,a \nonumber \\
&-b\dot{x}\,-k\, x \, +\,F_0 \,\cos\left( \omega_F \,t \right)\, = m\,\ddot{x} \nonumber \\
&-\frac{b}{m}\dot{x}\,-\,\frac{k}{m}\, x \, + \,\frac{F_0}{m} \,\cos\left( \omega_F \,t  \right)\, = \,\ddot{x},
\end{align}

La ecuación sale con un recuadro usando el comando "begin{important}[purple]".
Con el comando label se puede hacer mención a la ecuación en cualquier pare del texto, así: En la siguiente ecuación \ref{ecdifForA} ...
\begin{important}[purple]{align}
\ddot{x}\,+\,2\,\gamma\,\dot{x}+\,\omega_0^2 \, x \, =\,\frac{F_0}{m} \,\cos\left( \omega_F \,t \right). 
\label{ecdifForA}
\end{important}

Comando para insertar figura:
\begin{figure}[h]
\centering
\includegraphics[scale=0.5]{ForzadoFuerzas.png}
\end{figure}


\begin{align}
x(t)\,=\,x_t\,+\,x_e.
\end{align}



\begin{gather*}
\ddot{x}\,+\,2\,\gamma\,\dot{x}\,+\,\omega_0^2 \, x \, =0, 
\end{gather*}

Sistemas amortiguados:
\begin{important}[purple]{align}
x_t(t)\,&=\,C\,e^{-\gamma \,t} \cos\left(\omega \, t \,+\,\phi\right) &\text{subamortiguado} &&\nonumber \\
x_t(t)\,&=\,\left(C_1 \,+\,C_2\,t \right)\,e^{-\gamma\,t} &\text{críticamente amortiguado} &&\nonumber \\
x_t(t)\,&=\,C_1 \,e^{m'_1t}\,+\,C_2 \,e^{m'_2t}. &\text{sobreamortiguado} &&\nonumber 
\end{important}

Las ecuaciones dentro del texto se escribe usando los símbolos de peso: $x_h(t)$ o $x(t) = A \cos \left( \omega_0 t + \phi \right)$. 


Letras griegas
\begin{align}
    \alpha && \theta && \omega_0 && \omega_0^2
\end{align}

Fraccionarios
\begin{align}
   \left(  \frac{x}{y} \right)
\end{align}

Funciones trigonométricas
\begin{align}
 &   \cos \theta \\
 &   \tan \theta \\
 & \arctan \delta 
\end{align}


\begin{important}[purple]{gather}
x_e(t)\,=\,D\,\cos\left(\omega_F \, t \,-\,\delta \right), 
\end{important}


\begin{align*}
\dot{x}_e\,&=\,-D\,\omega_F\,\sen\left(\omega_F\,t-\delta \right) &\ddot{x}_e\,&=\,-D\,\omega_F^2\,\cos\left(\omega_F\,t-\delta \right).
\end{align*}
Sustituyendo:
\begin{align*}
&\left( -D\,\omega_F^2\,\cos\left(\omega_F\,t-\delta \right) \right) \,+\, 2 \gamma \left(-D\omega_F \sen \left( \omega_F \, t \,-\,\delta \right) \right)\,+ \, \omega_0^2 \left( D\,\cos\left(\omega_F\,t-\delta \right) \right)&& \\
&=\frac{F_0}{m}\,\cos\omega_F\,t, && \\
&\left(- D\,\omega_F^2\,+\,\omega_0^2\,D\right) \,\cos\left(\omega_F\,t-\delta \right) \,-\, 2 \gamma D\omega_F \sen \left( \omega_F \, t \,-\,\delta \right)  =\frac{F_0}{m}\,\cos\omega_F\,t, &&\\
&\left(- D\,\omega_F^2\,+\,\omega_0^2\,D\right)\left( \cos\omega_F\,t\, \cos \delta \, +\,\sen\omega_F\,t\, \sen \delta  \right)&&\\
&-\, 2 \gamma D\omega_F \left(\sen  \omega_F \, t \, \cos \delta - \sen \delta \cos \omega_F \, t\right) \,=\,\frac{F_0}{m}\, \cos\omega_F\,t.&&
\end{align*}
Igualando los coeficientes de los téminos $\sen\left( \omega_F \,t\right)$ y $\cos\left( \omega_F \,t\right)$:
\begin{align}
\sen\left( \omega_F \,t\right)&: D\left(\omega_0^2\,-\, \omega_F^2\right) \sen \delta  \,-\,2\gamma \,D\,\omega_F\,\cos\delta \,=\,0,  && 
\label{ecSenCos1}
\end{align} 
\begin{align}
\cos\left( \omega_F \,t\right)&: \,D\left(\omega_0^2\,-\, \omega_F^2\right) \cos \delta \,+\,2\gamma \,D\,\omega_F\,\sen\delta \,= \,\frac{F_0}{m}. &&
\label{ecSenCos2}
\end{align} 

De la ec. (\ref{ecSenCos1}): 
\begin{align*}
&D\left(\omega_0^2\,-\, \omega_F^2\right) \sen \delta  \,=\,2\gamma \,D\,\omega_F\,\cos\delta \nonumber && 
\end{align*}
\begin{important}[purple]{align}
&\tan\delta \,=\,\frac{2\gamma \,\omega_F}{\omega_0^2\,-\, \omega_F^2}.
\label{tand}
\end{important}	
Aquí vemos que la tangente de $\delta$ solo depende de las frecuencias y del coeficiente de amortiguamiento. Para encontrar $D$ de la ec \ref{ecSenCos2}, necesitamos expresiones del seno y coseno de $\delta$ en función de $\omega$ y $\gamma$.


\begin{important}[purple]{gather}
D\,=\frac{F_0/m}{\sqrt{\left(\omega_0^2\,-\, \omega_F^2\right)^2\,+\,\left( 2\gamma \,\omega_F\right)^2}} 
\label{amplitud}
\end{important} 

Resonancia:
\begin{important}[purple]{gather}
\omega_r =\omega_F = \sqrt{\omega_0^2 -2 \gamma^2},
\end{important}



Ejemplo con flechas en las ecuaciones:
\begin{align*}
&\omega_F \,=\, 0 &D &\, = \,\frac{F_0}{m \omega_0^2}\,=\,\frac{F_0}{k} && \\ 
&\omega_F\,=\, \omega_r &D& \, =\,D_{max}&& \\
&\omega_F\,=\, \omega_0 &D& \, =\,\frac{F_0}{2\,m\,\gamma \,\omega_0}&& \\
&\omega_F\rightarrow \infty &D& \, \rightarrow\,0&&
\end{align*}




Viñetas:
\begin{itemize}
\item La frecuencia de resonancia difiere de la frecuencia natural $\omega_0$ y de la frecuencia del oscilador amortiguado $\omega$.
\item La frecuencia de resonancia disminuye cuando el factor de amortiguamiento crece.
\item Para la luz láser ($\gamma$ pequeño) la curva de resonancia es aguda y alta, en cambio para la luz natural ($\gamma$ grande) la curva es ancha y baja.
\item Si b=0 la amplitud de resonancia es $\infty$.
\end{itemize}

\subsection{Ecuaciones de movimiento: resumen}
\label{ec total}
Ecuación de movimiento de un sistema forzado amortiguado :
\begin{important}[purple]{gather}
x(t)\,=\,x_t \, + \, D \cos \left(\omega_F \,t \,-\,\delta \right).
\end{important} 
$D$ y $\delta$ se encuentran a partir de  las ec. (\ref{amplitud}) y (\ref{tand}):

\begin{important}[purple]{gather*}
D\,=\frac{F_0/m}{\sqrt{\left(\omega_0^2\,-\, \omega_F^2\right)^2\,+\,\left( 2\gamma \,\omega_F\right)^2}}, 
\end{important} 
y 
\begin{important}[purple]{align*}
&\tan\delta \,=\,\frac{2\gamma \,\omega_F}{\omega_0^2\,-\, \omega_F^2}.
\label{tand}
\end{important}	
Las constantes que aparecen en $x_t$ se encuentran aplicando las condiciones iniciales a $x(t)$.


%\bibliographystyle{apsrev4-1}
%\bibliography{ultrathin,nematics}



\end{document}

